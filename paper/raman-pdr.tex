\RequirePackage{amsmath}
\documentclass[iop, apj]{emulateapj}
\usepackage[varg]{newtxmath}
\usepackage{newtxtext}
\usepackage[spanish,es-minimal,english]{babel}
\usepackage[utf8]{inputenc}
\usepackage{natbib}
\usepackage{microtype}
\usepackage{hyperref}
\usepackage{siunitx}
\bibliographystyle{apj}

\begin{document}
\title{Raman mapping of PDRs}
\author{William J. Henney}
\affil{%
  \foreignlanguage{spanish}{Instituto de Radioastronomía y
    Astrofísica, Universidad Nacional Autónoma de México, Apartado
    Postal 3-72, 58090 Morelia, Michaoacán, Mexico};
  w.henney@crya.unam.mx}

\begin{abstract}
  I show that the broad Raman-scattered wings of H\(\alpha\) can be used to
  map neutral gas illuminated by high-mass stars in star forming
  regions. The near wings (\(\Delta\lambda \approx \pm \SI{10}{\angstrom}\)) trace neutral columns Absorption features in the pseudo-continuum at 6634 and
  6663~\AA{} correspond to neutral oxygen far-ultraviolet absorption
  lines at 1027.43 and 1028.16~\AA{}.
\end{abstract}

\section{Introduction}
\label{sec:introduction}

\citet{Dopita:2016a} were the first to identify Raman scattering in the Orion Nebula.

\section{Observations}
\label{sec:observations}

\section{Discussion}
\label{sec:discussion}

The effective resolving power of the optical spectrograph is multiplied by 6.4 for the FUV domain. 

\bibliography{BibdeskLibrary}


\end{document}


\end{document}
%%% Local Variables:
%%% mode: latex
%%% TeX-master: t
%%% End:
